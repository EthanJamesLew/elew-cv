\documentclass{resume}

\usepackage[left=0.75in,top=0.6in,right=0.75in,bottom=0.6in]{geometry} % Document margins
\usepackage{url}

\usepackage[round]{natbib}  % bibliography package
\usepackage{bibentry}         %  full citation in the body of the text (turn off natbib if use it)
\nobibliography*   

\newcommand{\tab}[1]{\hspace{.2667\textwidth}\rlap{#1}}
\newcommand{\itab}[1]{\hspace{0em}\rlap{#1}}
\name{Ethan Lew} % Your name
\address{\url{https://eth0lew.com}} % Your address

\begin{document}

\begin{rSection}{Technical Skills}

\begin{tabular}{ @{} >{\bfseries}l @{\hspace{6ex}} l }
Research Experience \ & Formal Verification,  Cyber-physical Systems, Data-driven Controls \\ 
Programming & Python, R, Rust, C/C++, \LaTeX, Bash \\
Libraries/Frameworks & PyTorch, Tensorflow, Apache Spark, Gurobi, ONNX
\end{tabular}

\end{rSection}


\begin{rSection}{Work Experience}
\begin{rSubsection}{Galois, Inc., 421 SW 6th Ave, Portland, OR 97204}{2019-Present}{Research Engineer}{}
\item Excelled in developing research methodologies, writing sophisticated software prototypes, and exceeding project deliverables.
\item Contributed significant research and engineering to high-profile DARPA programs at the company: SSITH (2019-2021), Assured Autonomy (2019-2022), SDCPS (2021-2023), and Space-BACN (2022-Present).
\item Demonstrated expertise in system analysis and auditing, particularly in relation to the US Census Bureau's Disclosure Avoidance System (2021-Present).
\end{rSubsection}

\begin{rSubsection}{Johns Hopkins APL, 11100 Johns Hopkins Rd, Laurel, MD 20723}{2019}{Electrical Engineering Intern}{}
\item Research Projects: troposcatter communication system feasibility, FDTD solver for bodies of revolution, and methods and benchmarks for an adaptive radar resource manager (RRM).
\end{rSubsection}

\begin{rSubsection}{Summit Wireless Technologies, 20575 Von Neumann Dr., Beaverton, OR 97006}{2018-2019}{Electrical Engineering Intern}{}
\item Contributed engineering across several teams including the development of a RF power meter, a Linux wireless driver, and the wireless audio stack core. 
\end{rSubsection}

\begin{rSubsection}{Portland State University, 1825 SW Broadway, Portland, OR 97201}{2017}{Climate Research Intern}{}
\item Research Project: \textit{Sensitivity of Global Methane Bayesian Inversion to Surface Observation Data Sets and Chemical-Transport Model Resolution}.
\item Presented project at the Center for Climate and Aerosol Research (CCAR) symposium,
Council on Undergraduate Research (CUR) symposium, and the American Geophysical
Union (AGU) Fall Meeting.
\end{rSubsection}
\end{rSection}

\begin{rSection}{Education}

{\bf Portland State University, OR} \hfill {\em June 2016 - June 2019} 
\\ BS Electrical Engineering \quad \textit{Summa Cum Laude}\hfill {GPA: 4.00}
\\{\bf Portland Community College, OR} \hfill {\em March 2015 - June 2017} 
\\ Transfer Program \hfill {GPA: 4.00}

\end{rSection}

\begin{rSection}{Program Committees}

\begin{rSubsectionEmpty}{26th ACM International Conference on Hybrid Systems: Computation and Control}{2021}{Repeatability Evaluation Program Committee}{HSCC 2023}
\end{rSubsectionEmpty}

\begin{rSubsectionEmpty}{25th ACM International Conference on Hybrid Systems: Computation and Control}{2021}{Repeatability Evaluation Program Committee}{HSCC 2022}
\end{rSubsectionEmpty}


\begin{rSubsectionEmpty}{7th IFAC Conference on Analysis and Design of Hybrid Systems}{2021}{Repeatability Evaluation Program Committee}{ADHS 2021}
\end{rSubsectionEmpty}

\begin{rSubsectionEmpty}{24th ACM International Conference on Hybrid Systems: Computation and Control}{2021}{Repeatability Evaluation Program Committee}{HSCC 2021}
\end{rSubsectionEmpty}

\end{rSection}

\begin{rSection}{Publications}

\textbf{(authors alphabetical)} \bibentry{hekal2023falsify}

\bibentry{lew2023autokoopman}

\bibentry{davis2023leveraging}

\textbf{(authors alphabetical)} \bibentry{bak2022reachability}

\end{rSection}

\begin{rSection}{Awards}
\begin{rSubsectionEmpty}{Generalized RAcing Intelligence Competition (GRAIC)}{2022}{1st Place Head-to-Head Category}{CPS-IoT Week}\end{rSubsectionEmpty}

\begin{rSubsectionEmpty}{Electrical and Computer Engineering Capstone Poster Competition}{2019}{Best Overall Project}{PSU ECE}\end{rSubsectionEmpty}

\begin{rSubsectionEmpty}{Intel Compute Stick Challenge}{2016}{1st Place}{Intel}\end{rSubsectionEmpty}
\end{rSection}


\begin{rSection}{Projects}

\begin{rSubsection}{AutoKoopman: A Toolbox for Automated System Identification via Koopman Operator Linearization}{2022-Present}{}{Galois, Inc.}
\item AutoKoopman is a python library for learning Koopman operators for data-driven dynamical systems analysis and control. The library has convenient functions to learn systems using a few lines of code. It has a variety of linearization methods under shared class interfaces. These methods are pluggable into hyperparameter optimizers, which can automate the process of model optimization.
\end{rSubsection}

\begin{rSubsection}{Control System Analysis Framework (CSAF)}{2019-Present}{}{Galois, Inc.}
\item CSAF is a framework to minimize the effort required to evaluate, implement, and verify controller design (classical and learning enabled) with respect to the system dynamics.
\end{rSubsection}

\begin{rSubsection}{BESSPIN Automotive Demonstrator}{2020-2021}{}{Galois, Inc.}
\item The BESSPIN CyberPhysical Demonstrator was developed under the DARPA SSITH Program. The demonstrator is intended to showcase the use of the SSITH secure hardware technology in a passenger vehicle. Visitors will experience driving a vehicle as it is being hacked, and the process of hacking an unprotected and a SSITH protected car. 
\end{rSubsection}

\end{rSection}

\nobibliography{elew}
\bibliographystyle{apalike}

\end{document}