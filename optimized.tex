\documentclass{resume}

\usepackage[left=0.75in,top=0.6in,right=0.75in,bottom=0.6in]{geometry} % Document margins
\usepackage{url}

\usepackage[round]{natbib}  % bibliography package
\usepackage{bibentry}         %  full citation in the body of the text (turn off natbib if use it)
\nobibliography*   

\newcommand{\tab}[1]{\hspace{.2667\textwidth}\rlap{#1}}
\newcommand{\itab}[1]{\hspace{0em}\rlap{#1}}

\name{Ethan Lew} % Your name
\address{\url{https://eth0lew.com}} % Your address

\begin{document}

\begin{rSection}{Summary}
AI Research Scientist and Systems Engineer with 5+ years of experience developing autonomous systems and advanced AI solutions for high-profile DARPA and USAF-funded projects. Proven expertise in formal verification, machine learning, and cyber-physical systems—bridging cutting-edge research with real-world industry impact. Recognized for technical leadership, innovative methodologies, and award-winning contributions in autonomous systems.
\end{rSection}

\begin{rSection}{Technical Skills}

\begin{tabular}{ @{} >{\bfseries}l @{\hspace{6ex}} l }
Domain Expertise & Autonomous Systems, Formal Verification, AI/ML, Cyber-Physical Systems \\ 
Programming & Python, R, Rust, C/C++, \LaTeX, Bash \\
Libraries/Frameworks & PyTorch, TensorFlow, Apache Spark, scikit-learn, Gurobi, ONNX
\end{tabular}

\end{rSection}


\begin{rSection}{Work Experience}

\begin{rSubsection}{P-1.ai}{2024-Present}{Senior Systems Engineer}{}
\item Advancing Artificial General Engineering Intelligence by developing scalable, multi-modal AI frameworks for complex autonomous systems.
\end{rSubsection}

\begin{rSubsection}{Galois, Inc., 421 SW 6th Ave, Portland, OR 97204}{2019-2024}{Research Engineer}{}
\item Awarded a Phase I SBIR grant on vehicle autonomy and served as Principal Investigator (PI), developing an energy-aware mission planning framework for unmanned aerial systems.
\item Led the DARPA Assured Autonomy project by designing and implementing a Control Systems Analysis Framework (CSAF) that streamlined simulation and formal verification of AI-driven flight control systems.
\item Co-developed a hardware-assisted automotive cybersecurity demonstrator under DARPA’s SSITH program, showcasing secure processor designs that thwart cyberattacks on vehicle control units in live demonstrations.
\item Pioneered an integrated approach combining falsification and reachability analysis for validating autonomous control systems—achieving benchmark improvements and contributing to a first-author publication at ACM HSCC 2024.
\item Mentored a multidisciplinary team to design an autonomous blimp UAV prototype for an Office of Naval Research challenge, resulting in a successfully flight-tested model and an open-source design guide.
\end{rSubsection}

\begin{rSubsection}{Johns Hopkins APL, 11100 Johns Hopkins Rd, Laurel, MD 20723}{2019}{Electrical Engineering Intern}{}
\item Researched troposcatter communication system feasibility, developed an FDTD solver for bodies of revolution, and devised benchmarks for an adaptive radar resource manager (RRM).
\end{rSubsection}

\begin{rSubsection}{Summit Wireless Technologies, 20575 Von Neumann Dr., Beaverton, OR 97006}{2018-2019}{Electrical Engineering Intern}{}
\item Contributed across teams by developing an RF power meter, a Linux wireless driver, and core components of the wireless audio stack.
\end{rSubsection}

\begin{rSubsection}{Portland State University, 1825 SW Broadway, Portland, OR 97201}{2017}{Climate Research Intern}{}
\item Investigated the \textit{Sensitivity of Global Methane Bayesian Inversion to Surface Observation Data Sets and Chemical-Transport Model Resolution}.
\item Presented findings at the Center for Climate and Aerosol Research (CCAR), Council on Undergraduate Research (CUR), and American Geophysical Union (AGU) Fall Meeting.
\end{rSubsection}

\end{rSection}

\begin{rSection}{Education}

{\bf Portland State University, OR} \hfill {\em June 2016 - June 2019} 
\\ BS Electrical Engineering \quad \textit{Summa Cum Laude}\hfill {GPA: 4.00}
\\{\bf Portland Community College, OR} \hfill {\em March 2015 - June 2017} 
\\ Transfer Program \hfill {GPA: 4.00}

\end{rSection}

\begin{rSection}{Program Committees}

\begin{rSubsectionEmpty}{26th ACM International Conference on Hybrid Systems: Computation and Control}{2021}{Repeatability Evaluation Program Committee}{HSCC 2023}
\end{rSubsectionEmpty}

\begin{rSubsectionEmpty}{25th ACM International Conference on Hybrid Systems: Computation and Control}{2021}{Repeatability Evaluation Program Committee}{HSCC 2022}
\end{rSubsectionEmpty}

\begin{rSubsectionEmpty}{7th IFAC Conference on Analysis and Design of Hybrid Systems}{2021}{Repeatability Evaluation Program Committee}{ADHS 2021}
\end{rSubsectionEmpty}

\begin{rSubsectionEmpty}{24th ACM International Conference on Hybrid Systems: Computation and Control}{2021}{Repeatability Evaluation Program Committee}{HSCC 2021}
\end{rSubsectionEmpty}

\end{rSection}

\begin{rSection}{Peer-Reviewed Publications}

\textbf{(authors alphabetical)} \bibentry{10.1007/978-3-031-78750-8_12}

\bibentry{khandait2024arch}

\textbf{(authors alphabetical)} 
 \bibentry{abowd20232010}

\textbf{(authors alphabetical)} \bibentry{hekal2023falsify}

\bibentry{lahouel2024learning}

\bibentry{lew2023autokoopman}

\bibentry{davis2023leveraging}

\textbf{(authors alphabetical)} \bibentry{bak2022reachability}

\end{rSection}

\begin{rSection}{Awards}
\begin{rSubsectionEmpty}{Generalized RAcing Intelligence Competition (GRAIC)}{2022}{1st Place Head-to-Head Category}{CPS-IoT Week}
\end{rSubsectionEmpty}

\begin{rSubsectionEmpty}{Electrical and Computer Engineering Capstone Poster Competition}{2019}{Best Overall Project}{PSU ECE}
\end{rSubsectionEmpty}

\begin{rSubsectionEmpty}{Intel Compute Stick Challenge}{2016}{1st Place}{Intel}
\end{rSubsectionEmpty}
\end{rSection}

\begin{rSection}{Funded Projects}
\begin{rSubsection}{SBIR: Phase I: RHEIA-F: Robust High-fidelity Energy-Informed Autonomy Framework}{2024}{PI: \textbf{Ethan Lew}; Co-PI: Nicola Bezzo}{AFRL Funded, Galois Inc. Award: \$179,934}
\item Developed the Robust High-fidelity Energy-Informed Autonomy Framework (RHEIA-F) integrating comprehensive energy management into unmanned aerial systems (UAS) mission planning. This advanced framework is directly transferable to DoD embedded systems, autonomous vehicles, and space applications.
\end{rSubsection}
\end{rSection}

\nobibliography{elew}
\bibliographystyle{apalike}

\end{document}
